
While various parallel tools for compression gzip in parallel exist, no tool for parallel decompression of arbitrary gzip files is known to us.

\paragraph{Parallel Gzip Compression}

The \texttt{bgzip} tool~\cite{htslib} divides the data to be compressed into chunks, compresses those in parallel as independent gzip streams, adds metadata, and concatenates the gzip streams into a Blocked GNU Zip Format (BGZF) file, which itself is a valid gzip file.
Alternatively, \pigz~\cite{pigz} compresses chunks in parallel as separate \deflate streams instead of gzip streams.
It uses workarounds such as empty \rawblocks to avoid having to bit-shift the results to fit the bit alignment of the previous \deflate stream when concatenating them.
There also exists a multitude of hardware implementations to speed up gzip compression for example for compressing network data~\cite{gzip_on_a_chip,qat,Ledwon2020HighThroughputFH}.

\paragraph{Modified Gzip File Formats Enabling Parallel Decompression}

Parallelization of gzip decompression is more difficult because, firstly, the position of each block is only known after decompressing the previous block.
Secondly, to decompress a \deflate block, the last \SI{32}{\kibi\byte} of decompressed data of the previous \deflateblock have to be known.

Some solutions like BGZF~\cite{htslib} work around these two issues by adjusting the compression to limit the dependency-introducing backward pointers, limiting Huffman code bit alignments, or storing additional Huffman code bit boundaries~\cite{gpu-decompresion}.
BGZF not only contains multiple independent gzip streams but those gzip streams also contain metadata storing the compressed size of each gzip stream to allow skipping over them.
Some hardware~\cite{cmos-gzip-decompression} and GPU~\cite{gpu-huffman} implementations speculatively decode Huffman codes ahead in the stream.
These implementations make use of the self-synchronizing nature of Huffman codes and reach decompressed data bandwidths of \SI{2.6}{\giga\byte/\second}~\cite{cmos-gzip-decompression}.



\paragraph{Two-Stage Decoding}

The closest work and the one that this work builds upon is \pugz~\cite{pugz}.
It uses a two-stage decompression scheme, which has been summarized in \cref{sct:theory:two-stage-decompression} and has been implemented in \pragzip.
